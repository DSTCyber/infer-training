
\begin{frame}{Infer: An Introduction}
\begin{columns}
	\begin{column}{0.5\textwidth}
		Infer is
		\begin{itemize}
			\item a \textbf{static analysis} tool for \textbf{finding bugs} in software
			\item for C, C++, Objective C, and Java (Servers and Mobile Apps)
			\item on very large code bases
			\item using continuous integration processes
			\item with a reputation for finding \textit{thousands} of bugs a month
			\item and is used by Facebook, Uber, Spotify, ...
		\end{itemize}
	\end{column}
	\begin{column}{0.5\textwidth}  %%<--- here
		Timeline:
		\begin{itemize}
			\item \textbf{2000s}: Queen Mary University: Seperation Logic and Bi-Abduction - Cristiano Calcagno, Dino Distefano and Peter O'Hearn 
			\item \textbf{2009}: Monoidics startup
			\item \textbf{2013}: Facebook buys out Monoidics
			\item \textbf{2015}: Facebook open sources Infer
		\end{itemize}
	\end{column}
\end{columns}
\end{frame}

\begin{frame}{Infer: What makes it useful?}
	How does it do it?
	\begin{itemize}
		\item Compositional program analysis - analyzes procedures independently  to
		\item Produce logical summaries, using the theories of 
		\item \textbf{Abstract Interpretation}: "Inferbo"
		\\eg intervals domains, bitvector arithmetic domains, etc
		\item \textbf{Separation Logic} using Bi-abduction
		\item \textbf{Source-Sink Analysis} - "Quandary"
		\item \textbf{Linting} - "AL" 
	\end{itemize}
\end{frame}

\begin{frame}{Infer: How does it work?}

\textbf{Operations:}

\begin{itemize}
	\item Infer \textit{hooks} the \textit{compilation} process
	\item Processes the parsing phases: \\Parse Tree, AST, Use-Defs, Control Flow Graphs\\
	Type Information, Data Flow
	\item Parse data is analyzed, summarised and stored
	\item Then a set of \textbf{checkers} and \textbf{linters} are applied 
	\item produce bug reports
\end{itemize}

\end{frame}
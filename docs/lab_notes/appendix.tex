\appendix

\section{Installing Infer}


We will download and install Infer without using Docker. 
We assume you are using Ubuntu as your OS environment.

We are going to install Infer. Normally you would follow the instructions on Infer's 
``Getting Started'' website - \verb|https://fbinfer.com/docs/getting-started.html| - 
however in my experience the documented process never worked.

\dothis{Installing Infer}
\begin{enumerate}
	\itemsep0em 
	\item Download the version 0.17.0 tarball from Github:\\
	\verb|https://github.com/facebook/infer/releases|
	\item Extract it in a home directory \\
	\verb|# cd ~/bin && tar xvf infer-0.17.0.tar.gz |
	\item Add the Infer binary to the PATH environment in \verb|.bashrc|
	\begin{verbatim}
	PATH=~/bin/infer/bin:$PATH	
	export PATH
	\end{verbatim}
	\item Test that your installation runs. In a \textbf{new shell} run Infer - you should see the following
	\begin{verbatim}
	# infer
	Nothing to compile. Have you run `infer capture`? Try cleaning the build first.
	There was nothing to analyze.
	
	No issues found 	 
	\end{verbatim}
	\item Ensure that the Infer man pages are accessible by running the following command:\\
	\verb|# man infer|\\
	The man pages are very comprehensive but also somewhat confusing. Another source of
	information is the Infer documentation web page - \verb|https://fbinfer.com/docs/getting-started.html|. There are a number of important pages that cover more advanced operations. \\
\end{enumerate}

That completes the Infer installation. 

\section{Building the Audit Targets Natively}

We will use the following three Linux games to explore Infer.

\begin{itemize}
	\itemsep0em 
	\item \textbf{Angband} - \verb|https://rephial.org/|
	\item \textbf{Skynet} - \verb|https://github.com/cloudwu/skynet|
	\item \textbf{BSDGames} - \verb|https://github.com/vattam/BSDGames|
\end{itemize}

The targets were chosen because together they exhibit a good diversity of bugs
and have modest build and analysis requirements. 
Some targets because of size and complexity issues require large amounts of 
memory (i.e. $>$ 16 GB) and compute time (i.e. $>$ 10 minutes) which is not appropriate
for the typical resources available for this lab.

For the purposes of the lab we provide the source code tarballs required. The reason
for this is to make sure we are auditing the same fixed version of code with known bugs.

\subsubsection{Angband}

We are going to download, extract and build the game from source code.

\dothis{Build Angband - Part One}

\begin{enumerate}
	\itemsep0em
	\item Make an audit directory\\
	\verb|# mkdir ~/audit|
	\item Download and move the game into the audit directory- TODO 
	\item Extract it\\
	\verb|# tar xvf angband-4.2.0.tar.gz|
	\item Change directory to the root of the game repo	\\
	\verb|# cd angband-4.2.0|	
\end{enumerate}

At this point we need to investigate how to build the game.
Run the following command to examine what build options are available.

\vspace{0.5cm}

\note{In principle you would want to enable as many of the options that make sense in order to be able to analyse as much of the code base as possible.}

\vspace{0.5cm}
Examine the options available in building the code:

\begin{verbatim}
# ./configure --help 

...

Optional Features:
--disable-option-checking  ignore unrecognized --enable/--with options
--disable-FEATURE       do not include FEATURE (same as --enable-FEATURE=no)
--enable-FEATURE[=ARG]  include FEATURE [ARG=yes]
--enable-curses         Enables Curses frontend (default: enabled)
--enable-x11            Enables X11 frontend (default: enabled)
--enable-sdl2           Enables SDL2 frontend (default: disabled)
--enable-sdl            Enables SDL frontend (default: disabled)
--enable-win            Enables Windows frontend (default: disabled)
--enable-test           Enables test frontend (default: disabled)
--enable-stats          Enables stats frontend (default: disabled)
--enable-sdl2-mixer     Enables SDL2 mixer sound support (default: disabled
unless SDL2 enabled)
--enable-sdl-mixer      Enables SDL mixer sound support (default: disabled
unless SDL enabled)
--disable-ncursestest       Do not try to compile and run a test ncurses program
--disable-sdl2test       Do not try to compile and run a test SDL2 program
--disable-sdltest       Do not try to compile and run a test SDL program

Optional Packages:
--with-PACKAGE[=ARG]    use PACKAGE [ARG=yes]
--without-PACKAGE       do not use PACKAGE (same as --with-PACKAGE=no)
--with-setgid=NAME      install angband as group NAME
--with-private-dirs     use private scorefiles/savefiles
--with-no-install       don't install, just run in-place
--with-ncurses-prefix=PFX   Prefix where ncurses is installed (optional)
--with-ncurses-exec-prefix=PFX Exec prefix where ncurses is installed (optional)
--with-x                use the X Window System
--with-sdl2-prefix=PFX   Prefix where SDL2 is installed (optional)
--with-sdl2-exec-prefix=PFX Exec prefix where SDL2 is installed (optional)
--with-sdl-prefix=PFX   Prefix where SDL is installed (optional)
--with-sdl-exec-prefix=PFX Exec prefix where SDL is installed (optional)

\end{verbatim}
\dothis{Build Angband - Part Two}

\vspace{0.5cm}

Now we need to test that we can properly build the game such that Infer can analyse it:

\begin{enumerate}
	\item \todo{check - we need to make sure Clang and all the libraries are installed}
	\item\textbf{ Build Configuration}: In this case we will configure the build systems as follows:\\
	\verb|# CC=clang CXX=clang++ ./configure --enable-sdl2|
	\item \textbf{Build}: Make sure that Clang is being used to compile the code.\\
	\verb|# make| \\
	There should be no errors.
\end{enumerate}\

\subsubsection{Skynet}

We are going to download, extract and build the game from source code.

\dothis{Build Skynet}

\begin{enumerate}
	\itemsep0em
	\item Download and move the game into the audit directory- TODO 
	\item Extract it\\
	\verb|# tar xvf skynet.tar.gz|
	\item \todo{Install dependencies (detail ...)}
	\item Change directory to the root of the game repo	\\
	\verb|# cd skynet|	
	\item Build the code:\\
	\verb|# CC=clang CXX=clang++ make linux|
\end{enumerate}

Verify that the code was built without errors and that the compilation used Clang.

\subsubsection{BSDGames}

We are going to download, extract and build the game from source code.

\dothis{Build BSDGames}

\begin{enumerate}
	\itemsep0em
	\item Download and move the game into the audit directory- TODO 
	\item Extract it\\
	\verb|# tar xvf BSDgames.tar.gz|
	\item \todo{Install the library dependencies: ncurses, Lex, Yacc, libcrypto}
	\item Change directory to the root of the game repo	\\
	\verb|# cd BSDGames|	
	\item Run the build configuration tool:\\
	\verb|# ./configure|
	\item Edit the file \verb|./Makeconfig|\\
	Change the compilers from \verb|gcc/g++| to \verb|clang/clang++|\\
	Note: the usual method for specifying the compilers as environment variables does not seem effective.
	\item Build the code:\\
	\verb|# make |\\
	You \textbf{will} see compilation problems but this seems to be expected with this repo.
\end{enumerate}

Verify that the code was mostly built without errors and that the compilation used Clang.

\vspace{1cm}

At this point we have prepared a set of targets ready for analysis with Infer.
In practice you should not underestimate the effort required to properly prepare
the target build system for analysis. 
You would normally try to enable as many build options
as possible to maximise the amount of code that can be analysed. 
Depending on your objectives you may also want to build and analyse the library dependencies as well.


\section{Lab Three: Quandary}

Quandary is an experimental checker in Infer. 
It performs \textit{static taint analysis}, also called \textit{source-sink} analysis.
The essential idea is that you can specify a \textit{data-flow} between a
\begin{itemize}
	\item \textbf{source} function - that is the return value of the function;
	\item \textbf{sink} function: to a specific argument or any argument of that function.
\end{itemize} 

This is a really interesting checker from a security auditing perspective. 
For example, it allows you to specify a test from attacker-controlled data 
(for example, environment variables) to security sensitive functions (for example, \texttt{execve} functions).

For technical reasons this is also an impressive checker because it does this 
analysis \textit{statically} and efficiently.

\subsection{Running Quandary}

Like all checkers , we can run \textit{Quandary} as an additional analysis \textit{or} on its own.

\begin{itemize}
	\item As an \textbf{additional} checker: \\
	\verb|# infer run --quandary --keep-going -- make |
	\item \textbf{Inferbo only}:\\
	\verb|# infer run --quandary-only --keep-going -- make |	
\end{itemize}

\exercise{Working with Quandary - Part One}

\vspace{0.5cm}
For each of the targets run Quandary on its own and generate HTML bug reports.
Open each of the bug reports and examine the various bug \textit{classes}.
Start with \texttt{BSDGames} in this example.

\textbf{Remember: }
\begin{itemize}
	\item Delete any previous Infer data:\\
	\verb|# rm -rf ./infer-out| \\
	Otherwise the results of previous analysis may leak into the reports.
	\item Clean the build:\\
	\verb|# make clean|\\
	Otherwise Infer won't analyze anything.
\end{itemize}

You should see a summary like:

\begin{verbatim}
...
Summary of the reports

SHELL_INJECTION: 4
\end{verbatim}

By default, Quandary looks for data-flows between environment variables (as returned by \texttt{gentenv()})
and the \texttt{execve()} functions. These data-flows represent command injection vulnerabilities if there
is no sanitisation of the environment variable.

\exercise{Working with Quandary - Part Two}

\vspace{0.5cm}
In this example we are going to define our own source-sink data-flow we are interested in.

\begin{itemize}
	\item Create the following JSON file: \texttt{.inferconfig} in the root of the target 
	repo.
	\item Insert the following contents:\\
	\noindent\begin{verbatim}
    {
        "quandary-sources": [
        {
            "procedure": "getenv",
            "kind": "Logging"
        }
        ],
        "quandary-sinks": [
        {
            "procedure": "atoi",
            "kind": "Logging"
         }
         ]
    }	
	\end{verbatim}
	\item Re-run the Quandary analysis:\\
	\verb|# infer run --keep-going --quandary-only -- make |
\end{itemize}

\textbf{Remember: }
\begin{itemize}
	\item Delete any previous Infer data:\\
	\verb|# rm -rf ./infer-out| \\
	Otherwise the results of previous analysis may leak into the reports.
	\item Clean the build:\\
	\verb|# make clean|\\
	Otherwise Infer won't analyze anything.
\end{itemize}

You should see a summary like:

\begin{verbatim}
...
Summary of the reports

SHELL_INJECTION: 4
QUANDARY_TAINT_ERROR: 3
\end{verbatim}

In this example we have picked three examples of a data-flow between an environment variable
and a function we defined as interesting: \verb|atoi()|. This is an interesting flow because
the function \verb|atoi()| has no error handling and fails silently. 
If you click on the trace for these reports you should indeed see an actual data-flow analysis
that reveals a bug. In fact any use of the function \verb|atoi()| should be considered a bug
however when confronted with lots of bugs these ones found by Quandary should be prioritised.

It is also possible to be specific about \textit{which argument} of the sink function you want 
to trace. By default, a data-flow to \textit{any argument} of the sink function triggers a report.

\todo{show how to specify this flow}

\subsection{Review}

What have we learnt so far?

\begin{itemize}
	\item Quandary is a static taint tracing checker;
	\item Quandary finds by default potential \textit{command injection} vulnerabilities;
	\item How to create our own source-sink analyses using the JSON file \verb|.inferconfig|
\end{itemize}


\subsection{What Next?}

In the next section we are going to learn how to use Infer's \textit{linting} capabilities.
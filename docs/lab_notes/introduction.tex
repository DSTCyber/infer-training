\section{Introduction}

\subsection{Overview}

\textit{Infer} is a static analysis tool whose purpose
is to find bugs in code that normal code compilation won't find.
Infer is owned by Facebook who open-sourced the tool in 2015.
Since then the tool has been used by quite a few Internet scale companies
such as Uber, Spotify, Mozilla as well as Facebook to find thousands of bugs 
a month in their code bases. It has a good reputation.

\textit{Infer} has a significant foundation in advanced program analysis techniques
developed by academia. This includes theories such as \textit{Abstract Interpretation} and 
\textit{Separation Logic}. You do not need to understand these theories to use the tool
but they are the reason why the tool is effective in finding non-trivial bugs 
that the compiler doesn't catch. The cost however for this ability is that the
tool can take considerable resources to run in terms of memory and compute power.

\textit{Infer} can analyse C, C++, Objective-C and Java code. Not surprisingly
these are languages that matter a lot to Facebook: mobile applications and backend
server systems. In this training we focus on C and C++ code-bases. To be able to analyse
code with Infer you must be able to compile and build it. The reason for this is
that Infer captures many of the front-end compilation data-structures in order
for it to perform the advanced analyses it applies.

The bug reports generated by a static analysis tool such as Infer differ from 
the bugs found by other techniques from dynamic analysis. With dynamic analysis
techniques such as fuzzing, the bugs mostly are true positives. 
The reason is simple: an actual execution that results in a failure reveals
that the bug is not just theoretically possible but actually exists.

In contrast static analysis attempts to consider not just the space of \textit{actual} executions
but the much larger space of \textit{all possible} executions.
However to make this computation feasible static analysis techniques use
\textbf{over-approximation strategies} to reduce the size of the analysis space. 
The price you pay for computability is a reduction in the precision of the
analysis. It is important to understand this when inspecting Infer's bug reports.

\subsection{Learning Outcomes}

Upon completion of this course, the student will be able to:
\begin{itemize}
	\itemsep0em
\item Understand static analysis and its applications and limitations to bug finding
\item Install Infer on a Linux machine
\item Run Infer against a C/C++ source code targets
\item Enable different Infer checkers, including \textit{Inferbo} and \textit{Quandary}
\item Generate Infer HTML and JSON bug reports
\item Read and understand Infer bug reports for code audit purposes
\item Understand and develop Infer linters for syntactic bug finding using Infer's AL language 
\item How to generate Call Flow Graphs (CFGs) and other front end compilation data for target code understanding and debugging
\item How to use Infer with CMake using a \textit{compilation database}
\end{itemize}

\subsection{Pre-requisites}
\begin{itemize}
\item Knowledge and some proficiency in C and C++ coding, compilation and debugging
\item Knowledge of C/C++ bug classes and Code Review principles and practice 
\item Knowledge of Linux Shell commands and operating system operation
\end{itemize}

\subsection{Lab Exercises}

\begin{itemize}
\item \textbf{Lab ONE}: Basic analysis of a target
\item \textbf{Lab TWO}: \textit{Inferbo}: Buffer Overflow and Integer Bugs
\item \textbf{Lab THREE}: \textit{Quandary}: \\Static Taint Analysis and Command Injection Bugs
\item \textbf{Lab FOUR}: \textit{Linters}: How to write a linter in AL
\end{itemize}
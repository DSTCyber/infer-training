\section{Lab Two: Inferbo}

The purpose of this lab is to use Infer to find:
\begin{itemize}
	\itemsep0em
	\item \textbf{Integer Bugs} - overflows, sign vs unsigned bugs, sign extension bugs etc;
	\item \textbf{Buffer Overflows}: memory corruption bugs	
\end{itemize}

Infer uses an experimental checker called ``Inferbo'' to do this analysis. 
This checker uses Abstract Interpretation Theory to analyse the effects of integer expressions
and their application to memory structures such as arrays.

\subsection{Running Inferbo}

Like all checkers , we can run \textit{Inferbo} as an additional analysis \textit{or} on its own.

\begin{itemize}
	\item As an \textbf{additional} checker: \\
	\verb|# infer run --bufferoverun -- make |
	\item \textbf{Inferbo only}:\\
	\verb|# infer run --bufferoverrun-only -- make |	
\end{itemize}

\exercise{Working with Inferbo}

\vspace{0.5cm}
For each of the targets run Inferbo on its own and generate HTML bug reports.
Open each of the bug reports and examine the various bug \textit{classes}.

\textbf{Remember: }
\begin{itemize}
	\item Delete any previous Infer data:\\
	\verb|# rm -rf ./infer-out| \\
	Otherwise the results of previous analysis may leak into the reports.
	\item Clean the build:\\
	\verb|# make clean|\\
    Otherwise Infer won't analyze anything.
\end{itemize}

You should see a summary like:
\begin{verbatim}
	...	
	
	Summary of the reports
	
	BUFFER_OVERRUN_L3: 14
	BUFFER_OVERRUN_L2: 9
	INFERBO_ALLOC_MAY_BE_BIG: 9
	INTEGER_OVERFLOW_L2: 8
	INFERBO_ALLOC_IS_ZERO: 4
	INFERBO_ALLOC_MAY_BE_NEGATIVE: 1
	INTEGER_OVERFLOW_R2: 1
	BUFFER_OVERRUN_L1: 1	
\end{verbatim}

Notice that the various bug classes have suffixes like ``L1'' and ``L3''. 
I'm not exactly sure what they mean but I strongly suspect that they relate 
to specific class variations of a bug that is being reported.

A few specific things to know:
\begin{itemize}
	\itemsep0em
	\item \textbf{Understanding Traces}: The first few lines of a \verb|BUFFER_OVERRUN| trace are very important. It tells you why it thinks there is a memory error. For example:\\
	\verb|Offset: [3, 18] Size: 16| is interpreted as the buffer is sized 16 but the index range is predicted by analysis to vary between 3 and 18. 
	\item \textbf{Confusing arithmetic}: Some maths expressions seem to confuse Infer and therefore
	the range calculations seem to be in error. In particular bit operations and union structures seem a problem.
	\item \textbf{Precision}: Remember this is \textit{static} analysis. Runtime execution may
	be protective of the problem identified. However, a lot of bug reports would be eliminated if
	the source code had proper bounds checking in place.
\end{itemize}

In particular examine the following interesting \textit{specific} bugs:
\begin{itemize}
	\item See whiteboard :-)
\end{itemize}

\subsection{What have we learnt so far?}

In my experience the Inferbo reports are always worth thinking through. 
They often can be eliminated with simple bounds checking at the identified problem site.

\subsection{What Next?}

The next section we will look at \textit{Quandary} - a static source-sink or taint checker.


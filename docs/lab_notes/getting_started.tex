\section{Getting Started}

\subsection{Infer Installation Using Docker}

This section we are going to install Infer and the three audit targets
using Docker.
The purpose of using Docker is to avoid spending unnecessary time 
on resolving configuration and dependency issues with the tool and 
the various targets.
However in the appendix we show how to install and configure Infer 
natively for a Linux system.

These notes are on how to use Docker to do this lab. 

\dothis{Install and Run the Infer Docker}


\noindent From the root of the lab repo: 

\begin{verbatim}
# Create our lab directory to share files with host
mkdir ./lab

# Install and configure Docker on Ubuntu
sudo apt install docker.io
sudo usermod -aG docker ${USER}
su - ${USER}
git clone https://github.com/smagrath/infer-training
cd infer-training

# Build a docker image from the Dockerfile in sub-dir ```lab```
# Tag it infer-training 
docker build -t infer-training .

# Create a new container based on the image above 
# mount the host sub-dir ```content``` into the container 
docker create -it -v `pwd`/lab:/root/lab --name infer infer-training

# Boot the container
docker start infer

# Run a terminal in the container
docker exec -it infer /bin/bash

# In the Docker terminal run the configure script.
# This script pulls down and "patches" the audit targets.
cd lab
./configure.sh
\end{verbatim}

You now have a running Docker container with Infer and the three audit targets
installed and configured along with their dependencies. 
Also, the host and the Docker container share a common sub-directory on
which allows files to be transferred to the host machine.
This shared directory is the \verb|./lab| directory of this training repo. 
The main reason for doing this is so we can use the host OS's web browser
to view the Infer reports and other data files.
The Docker image has no GUI support.

When you're finished with your Docker image complete the following:

\begin{verbatim}
# docker stop infer
# docker rm infer
# docker rmi infer-training
\end{verbatim}

This will stop the Docker container and delete the Infer image from the host machine.

\section{Audit Targets}

We will use the following three Linux games to explore Infer.

\begin{itemize}
	\itemsep0em 
	\item \textbf{Angband} - \verb|https://rephial.org/|
	\item \textbf{Skynet} - \verb|https://github.com/cloudwu/skynet|
	\item \textbf{BSDGames} - \verb|https://github.com/vattam/BSDGames|
\end{itemize}

The targets were chosen because together they exhibit a good diversity of bugs
and have modest build and analysis requirements. 
Some targets because of size and complexity issues require large amounts of 
memory (i.e. more than 16 GB) and compute time (i.e. more than 10 minutes) which is not appropriate
for the typical resources available for this lab.

The Docker container has the build systems, source code and library dependencies
already installed. 
However we have included in the Appendix the process to build the targets 
from scratch for the purpose of being analysed by Infer.

Infer uses Clang and therefore the targets must be buildable with Clang.
Moreover Infer hooks several build tools such as \verb|make|, however
it can take a bit of effort to get Infer to work with tools like \verb|CMake|.
We look at how this down in TODO.

In practice you should not under-estimate how much effort is required to 
properly prepare a target for Infer. Your goal is to maximise analysis
coverage which means that you need to configure the target build system 
to use as many features and libraries as possible. 
Recursively, you should consider analysing the source code of the library dependencies
as well.

In the following subsections we will complete the build configuration for the
targets.
We could have scripted this work however this is not helpful to you.
Part of using Infer effectively is understanding and controlling the build
process of the audit targets intimately.

\subsection{Angband}

In the Docker terminal:

\dothis{Verify you can build Angband}


\begin{verbatim}
$ cd angband-4.2.0
$ CC=clang CXX=clang++ ./configure --enable-sdl2
$ make
\end{verbatim}

The build should complete successfully with no errors.

\subsection{Skynet}

We are going to build \verb|Skynet|. 
I've always wanted to build Skynet.
I've always wanted to ride a Harley-Davidson, wear a leather jacket and 
wield a sawn-off shotgun. I look cool in sun-glasses too.

To build Skynet we need to explicitly identify that it is a \verb|linux| 
build configuration. You will need to remember this throughout the course.

In the Docker terminal:

\dothis{Verify you can build Skynet}

\begin{verbatim}
$ cd skynet
$ CC=clang CXX=clang++ make linux
\end{verbatim}

The build should complete successfully with no errors.

\subsection{BSDGames}

We will need to manually edit (using Vim) the \verb|Makeconfig| file to use Clang instead of GCC.
The usual method of setting environment variables is ineffective.

In the Docker terminal:

\dothis{Verify you can build BSDGames}

\begin{verbatim}
$ cd bsdgames
$ ./configure
$ vim ./Makeconfig
# --> change the compilers from GCC/G++ to Clang/Clang++
$ make
\end{verbatim}

The build should complete successfully with no errors however you will 
see plenty of warnings. It's old code.